\documentclass[11pt]{gsasthesis} % 10,11 and 12pt fonts allowed

%%%%%%%%%%%%%%%% PACKAGES YOU PROBABLY WANT %%%%%%%%%%%%%%%%
% Include packages you want. The gsasthesis style file already includes
% packages "setspace" and "tocbibind".

% Included in original document
\usepackage{etex} % extend the number of registers

% GSAS: "all margins should be at least 1 inch."
\usepackage[margin={1.0in}]{geometry}
% If you want asymmetric margins for two-sided documents, use the "twoside" option, as in
% \usepackage[top=1in,bottom=1.5in,left=1in,right=1.5in,twoside]{geometry} 
% The left and right options become inner and outer margins 
% The default horizontal latex margin ratio is 2:3. The default vertical top:bottom margin ratio is 2:3 also. 
% You can also set it directly by passing the hmarginratio option to the geometry package, as in
% \usepackage[top=1in,left=1in,vmarginratio=2:3,hmarginratio=2:5,twoside]{geometry}

% Appendix package. Not necessary, but it does make managing appendices easier
\usepackage[titletoc]{appendix}

%%%%%%%%%%%%%%%% PACKAGES MAY WANT %%%%%%%%%%%%%%%%

% sideways tables and figures
\usepackage{rotating}

% tables that spill over multiple pages
\usepackage{longtable}

% references
% \usepackage{natbib}

% fonts that are nicer than defaults
\usepackage[sc]{mathpazo}
\usepackage{courier}

% Use 8-bit encoding that has 256 glyphs, pretty please
\usepackage[utf8]{inputenc}
\usepackage[T1]{fontenc}

% babel is required for blindtext, which generates random text
\usepackage[english]{babel}
\usepackage{blindtext}

% Slightly tweak font spacing for aesthetics
\usepackage{microtype}

% You need the footmisc package with the stable option if you want to have footnotes inside section titles, 
% for example to say that a particular chapter has been co-authored with someone. 
% The multiple option ensures that there is a comma between two consecutive footnotes
\usepackage[stable,multiple]{footmisc}

% Nicer captions
\RequirePackage[font=small,format=plain,labelfont=bf,textfont=it]{caption}
\addtolength{\abovecaptionskip}{1ex}
\addtolength{\belowcaptionskip}{1ex}

%%%%%%%%%%%%%%%% MSE Packages %%%%%%%%%%%%%%%%
% Base packages
\usepackage{fancyhdr}
\usepackage{amsmath}
\usepackage{amssymb}
\usepackage{geometry}
\usepackage{amsfonts}
\usepackage{mathtools}
\usepackage{setspace}
\usepackage{makecell}
\usepackage{amstext}
\usepackage{framed}
\usepackage{caption}
\usepackage{nicefrac}
\usepackage{xcolor}
\usepackage{listings}
\usepackage{verbatim}
\usepackage{soul}
\usepackage{gensymb}

% Inline Graphics
\usepackage{graphicx,float,subfig, grffile}
\graphicspath{{../figs/}{../figs/web/}{../figs/integration_test/planets/}}

% Captions
\captionsetup[figure]{font=small}

% MSE Terminal environment for screenshots
\lstdefinestyle{Terminal}
{
    basicstyle=\fontsize{8}{10}\color{black}\ttfamily
}

% MSE Code snippets
\lstdefinestyle{CodeSnippet}
{
    basicstyle=\fontsize{10}{12}\color{black}\ttfamily
}
% MSE Macros
\newcommand{\tty}[1]{\texttt{#1}}

% Hypertext References
\usepackage{hyperref,color,textcomp}
\definecolor{webgreen}{rgb}{0,.35,0}
\definecolor{webbrown}{rgb}{.6,0,0}
\definecolor{RoyalBlue}{rgb}{0,0,0.9}
\definecolor{purp}{rgb}{0.6,0.3,0.9}
\hypersetup{
   colorlinks=true, linktocpage=true, pdfstartpage=3, pdfstartview=FitV,
   breaklinks=true, pdfpagemode=UseNone, pageanchor=true, pdfpagemode=UseOutlines,
   plainpages=false, bookmarksnumbered, bookmarksopen=true, bookmarksopenlevel=1,
   hypertexnames=true, pdfhighlight=/O,
   urlcolor=webbrown, linkcolor=RoyalBlue, citecolor=webgreen,
   pdfauthor={Michael S. Emanuel},
   pdfsubject={Harvard IACS Masters Thesis (May 2020)},
   pdfkeywords={},
   pdfcreator={pdfLaTeX},
   pdfproducer={LaTeX with hyperref}
}
\hypersetup{pdftitle={Kepler's Sieve}}


%%%%%%%%%%%%%%%% COMPULSORY FIELDS %%%%%%%%%%%%%%%%

\title{Kepler's Sieve: \\Learning Asteroid Orbits from Telescopic Observations} % needs to match title on DAC
\author{Michael S. Emanuel} % full name as it appears on your GSAS record, needs to match name on DAC
\degreename{Master of Science}
\degreefield{Data Science} % Official name of subject as listed in GSAS handbook
\department{Institute of Applied Computational Science} % official name of department
\degreemonth{May} % Month of Defense (i.e. month when DAC was signed)
\degreeyear{2020} % Year the DAC was signed
\principaladvisor{Professor Pavlos Protopapas}

% Optionally, you can add a second advisor, but you can't have three
\secondadvisor{Professor Christopher H. Rycroft}

% Macro definitions
\newcommand{\N}{\mathbb{N}}
\newcommand{\Z}{\mathbb{Z}}
\newcommand{\Q}{\mathbb{Q}}
\newcommand{\R}{\mathbb{R}}
\newcommand{\B}{\mathbb{B}}
\newcommand{\mcL}{\mathcal{L}}
\newcommand{\hamiltonian}{\mathcal{H}}
\newcommand{\p}{\partial}
\renewcommand{\vec}[1]{\mathbf{#1}}
\newcommand{\xvec}{\vec{x}}
\newcommand{\uvec}{\vec{u}}
\newcommand{\bvec}{\vec{b}}

% MSE command to mark TODO items
\newcommand\todo[1]{\textbf{\textcolor{red}{#1}}}

\begin{document}

%%%%%%%%%%%%%%%% FRONTMATTER %%%%%%%%%%%%%%%%

\pagenumbering{roman} % GSAS wants roman page numbers for frontmatter

% the following four pages are required in that order. The first two pages are
% not allowed to have page numbers, this is taken care of in the class file.
\thesistitlepage
\copyrightpage
\begin{abstract}
% An abstract should be less than 350 words.
A novel method is presented to learn the orbits of asteroids from a large data set of telescopic observations.
The problem is formulated as a search over the six dimensional space of Keplerian orbital elements.
Candidate orbital elements are initialized by randomly choosing the six elements independently 
at random, either by sampling from a known asteroid or from uniformly distributed angles.
A statistical model describes the distribution of angular distances between the directions 
of observed detections to the predicted direction from the observatory site to a body with these orbital elements.
This model yields a log likelihood function, which attains large positive values when the candidate orbital elements are near 
to the elements of a real asteroid viewed many times in the data.
The candidate elements and the parameters describing the mixture distribution are jointly optimized using gradient descent.
Computations are performed quickly and efficiently on GPUs using the TensorFlow library.

The methodology of predicting the directions of telescopic detections is validated by demonstrating 
that out of approximately 5.69 million observations from the ZTF dataset,
3.75 million (65.71\%) fall within 2.0 arc seconds of the predicted directions of known asteroids.
The search process is validated in multiple stages.
First, I demonstrate that given an initial guess equal to a perturbation applied to the elements of asteroids present in the data,
the search process is able to successfully recover the true orbital elements to a high degree of precision.
Next, I demonstrate that a random initialization is able to converge to elements matching some known asteroids to high precision.
Finally, I use the search on observations that do not match any known asteroids.
I present orbital elements for [5] new, previously unknown asteroids.\\
\todo{Exact number of new asteroids presented}\\
All code for this project is publicly available on GitHub at \href{https://github.com/memanuel/kepler-sieve}{github.com/memanuel/kepler-sieve}.

\end{abstract}

% Center headings for table of contents, LOT, and LOF and make them smaller so that "Abstract", "Acknowledgments" and "Contents" all look alike. 
% Comment out if you want the default. If you want more control, use the "tocloft" package.
\renewcommand{\contentsname}{\protect\centering\protect\Large Contents}
\renewcommand{\listtablename}{\protect\centering\protect\Large List of Tables}
\renewcommand{\listfigurename}{\protect\centering\protect\Large List of Figures}

\tableofcontents % Table of contents

% The rest of the front matter: Lists of tables, figures, dedication and
% acknowledment is optional. Comment out whatever you don't like
% \listoftables
% \listoffigures
\begin{acknowledgments}
I would like to thank my advisor, Pavlos Protopapas, for suggesting this topic and for his consistent support, advice and encouragement. \\
I would like to thank Chris Rycroft, my secondary advisor, for guiding me through a paper in Applied Math 225 in which I explored
numerical integrators for solving solar system orbits.\\
I would like to thank Matt Holman and Matt Payne from the Center for Astrophics (CFA) for their advice on state of the art solar system integrators.\\
Most importantly, I would like to thank my wife Christie and my children Victor and Ren\'ee for their love and support.
The Covid-19 crisis struck just as my work on this thesis kicked into high gear.
It would not have been possible to complete it without extraordinary understanding and support from them.
\end{acknowledgments}
% \begin{dedication}
%   To my parents
% \end{dedication}


%%%%%%%%%%%%%%%% MAIN BODY %%%%%%%%%%%%%%%%
\pagenumbering{arabic} % reset page numbering and switch to arabic

% Introductory chapter. Comment out if you don't have an intro chapter, but I
% think most committees expect you to have one.
% Don't number the intro chapter, but add to to the table of contents
\addcontentsline{toc}{chapter}{Introduction}
\chapter*{Introduction}\label{ch:intro}
Determining the orbits of asteroids is one of the oldest problems in astronomy.
Classical methods are based on taking multiple observations of the same body through a telescope.
For an object that is large and bright enough, the human eye can ascertain the continuity of the motion,
i.e. that the data are multiple sightings of the same object.
Once enough sightings have been obtained, orbital elements can be solved using traditional
numerical methods, such as a least squares fitting procedure that seeks elements to minimize
the sum of squares error to all of the observations.

State of the art techniques for solving this problem are remarkably similar in spirit to the classical method.
Indeed, the first interstellar object, `Oumuamua, was discovered when astronomer Robert Weryk
saw it in images captured by the Pan-STARRS1 telescope on Maui.
\footnote{\href{https://en.wikipedia.org/wiki/\%CA\%BBOumuamua}{Wikipedia - Oumuamua}}
\footnote{\href{https://www.nytimes.com/2017/10/27/science/interstellar-object-solar-system.html}{NY Times - Astronomers Race to Study a Mystery Object from Outside the Solar System}}
More automated methods also exist.
Still, these methods are based on a search in the space of the observable data attributes: the time of observation (MJD), the right ascension (RA) and declination (DEC).
The apparent magnitude or brightness (MAG) is the third important observed quantity available for telescopic detections.
Two observations made close together in time at two points in the sky very near to each other have a relatively high probability of belonging to the same object.
Such a pair of observations is called a ``tracklet.''
Today's most automated approaches to identifying new asteroids from telescopic data are based on performing a greedy search of the observed data to extend tracklets.
Once a tracklet is identified, the algorithm attempts to extrapolate the path where future detections of this object might be.
After enough detections are strung together, a fitting procedure is tried to determine the orbital elements.

This is a solid technique and I do not mean to cast aspersion on it.
In this paper, however, I propose a new method which I believe has some significant advantages.
Rather than searching in the space of the data, i.e. (MJD, RA, REC, MAG), I propose to instead search over the six dimensional space of 
Keplerian orbital elements $(a, e, i, \Omega, \omega, f)$.
Why should we complicate things by searching implicitly, as it were, on the space of possible orbits,
rather than the simpler and more direct method currently used?
The main reason is to avoid a combinatorial explosion.

If you limit your search to candidate tracklets where you detect the same object multiple times in a short span of time,
you are going to miss any object that you detect only once or twice on a given night of observations.
But if this same object were seen on multiple nights, possibly separated over multiple days or longer, 
it becomes very costly to propose enough candidate tracklets to pick them up.
Indeed, you will soon face a combinatorial explosion in the number of possible tracklets.
A simplified model of the number of tracklets might be that we have a data set containing 
observations with a uniform density $\rho$ per day per degree squared of sky,
and we set a threshold $\tau$ in time and $\Delta$ in angular distance for how close a second observation must be to mark it as a candidate tracklet.

Here is a simple calculation showing the quadratic cost of enumerating candidate tracklets. \\
If you extend this further to tracks with 3 observations, the scaling gets even worse (cubic).\\
Let $\rho$ be the average density of detections per day per degree of sky.\\
Let $T$ be the number of days of observations in our data set.\\
Let $\tau$ be the threshold in days for 2 observations to be considered close enough in time to form a candidate tracklet.\\
Let $\Delta$ be the threshold angular distance in degrees for 2 observations to be considered close in the sky to form a candidate tracklet.\\
Let $A = 41,253$ be the number of square degrees in the sky.
\footnote{\href{https://en.wikipedia.org/wiki/Square_degree}{Wikipedia - Square Degrees in the Sky}}\\
Let $N = T \cdot A \cdot \rho $ be the total number of detections in the data set.\\
Let $\displaystyle{m = \tau \cdot \pi \Delta^2 \cdot \rho}$ be the average number of observations that will be close enough
to each candidate starting point of a tracklet.\\
Let $\displaystyle{NT_{2} = \frac{N \cdot m}{2!} = (T \cdot A) \cdot (\tau \cdot \pi \Delta^2) \cdot \frac{\rho^{2}}{2!} }$ 
be the number of candidate trackets of size $2$. \\
Let $\displaystyle{NT_{k} = \frac{N \cdot m^{k-1}}{k!} = (T \cdot A) \cdot (\tau \cdot \pi \Delta^2)^{k-1} \cdot \frac{\rho^{k}}{k!} }$ 
be the number of candidate trackets of size $k$. \\
We can see the bad news right away.
The number of tracklets of size $k$, $NT_{k}$, scales as $\rho^{k}$, $\tau^{k-1}$ and $\Delta^{2(k-1)}$.
Increasing the breadth of our tracklet search rapidly becomes prohibitively costly.

This is the principal motivation for searching in the space of orbital elements.
While it's a large 6 dimensional space, its size is fixed in relation to the amount of data we have.
To be more precise, the number of candidate orbital elements will scale with the number of \textbf{asteroids} $K$ we are trying to detect
rather than the much larger number of \textbf{detections} $N$ in our data set.
The cost of the search algorithm presented below scales linearly in the observation density $\rho$ for each candidate element analyzed.
The cost of the entire algorithm is therefore on the order of $N \cdot K$, with no explosion as you consider tracklets larger than 2.
The second major reason for searching in the space of orbital elements is that it permits the search algorithm to string together observations made far apart in time.
This is a capability that eludes searches based on tracklets.

I summarize here the key steps in the search algorithm.
The first step is to generate a set of candidate orbital elements.
This is done with a very simple approach, one which can almost certainly be improved on later: random initialization.
For four of the orbital elements, $a$, $e$, $i$, and $\Omega$, one of the $733,489$ catalogued asteroids is selected at random.
Its orbital elements are used to populate these four.
It is worth emphasizing that each element is initialized with an \textit{independent} random asteroid;
the four elements in this part will almost never match one of the known asteroids across all four elements.
The remaining two orbital elements, $M$ (mean anomaly) and $\omega$ (argument of periapsis), 
are sampled uniformly at random on the circle $[0, 2 \pi)$.
These are then converted to the representation using $(a, e, i, \Omega, \omega, f)$.

Once the candidate elements have been initialized, they are integrated numerically using the \tty{REBOUND} library.
This is considered to be the gold standard of their true orbits.
This initial integration is then used to filter the data set of ZTF observations to a subset that are relevant for searching for orbits.
A routine computes the direction $\uvec_{\mathrm{pred}}$ in the barycentric mean ecliptic (BME) reference frame
that an observer at a given observatory site on Earth would have seen light leaving an object with the candidate elements at a given observation time (MJD).
This direction is computed at each unique observation time in the ZTF data set.
A separate computation is performed once on all of the ZTF observations converting the observed triplets $(MJD, RA, DEC)$
into vectors $\uvec_{\mathrm{obs}}$, the direction of the observation in the BME frame.
The angular distance between the predicted and observed direction is computed.
A threshold (2.0 degrees) is applied, and all ZTF observations falling within this threshold are cached in memory of the search class.

During the main body of the search process, the elements will be adjusted by a small amount in each training round.
These perturbed elements will have their orbits evaluated using the Kepler two body model.
An implementation is performed on the GPU using TensorFlow that is fast and differentiable.
The ``ground truth'' numerically integrated orbit is used to provide an adjustment term 
so that the predicted orbits will match the true orbits exactly when the perturbation is zero.
The predicted orbit can therefore be considered to be a linearization of the true orbits based on the Kepler model.

The objective of the optimization function is based on the log likelihood of a statistical model for the 
distribution of distances between predicted and observed directions.
A lemma will demonstrate that for directions uniformly distributed on the sphere, the squared distance over the 
threshold distance would be uniformly distributed on the interval $[0, 1]$.
A mixture model is formulated, where the distance between every predicted and observed direction is modeled as a mixture of hits and misses.
The misses are distributed uniformly on $[0, 1]$.
The hits are distributed as a truncated exponential distribution.
The decay parameter $\lambda$ of this exponential process is associated with a resolution parameter $R$.
This model is equivalent to assuming that some fraction $h$ (for hits) of the detections are due to 
a real body with the candidate elements, and that the results of the detection will be normally distributed 
with a precision parameter equal to the resolution.
During the search process, the threshold parameter is also updated.
This dynamic threshold should not be confused with the original threshold of 2.0 degrees used to build the filtered training data.

The optimization process jointly optimizes the candidate orbital elements and three parameters in the mixture model:
the assumed number of hits $N_{h}$, the resolution $R$, and the threshold $\tau$. 
Intuitively, we want the model to gradually tighten its focus, and adjust the orbital elements so they hit as many observations as closely as possible.
But we \textit{don't} want the model to get ``faked out'' by trying to get the elements closer to observations that belong to \textit{other} asteroids.
The model needs some way to update probabilities that each observation is a hit or a miss, which it does using the mixture model.
Early on, the optimization will try to get close to the central tendency of the data set.
If the initialization was good, it will gradually tighten in the resolution and threshold parameters.
The gradients will encourage the model to adjust the candidate orbital elements so that some of the observations,
the ones it sees as highly probable hits, will be very close to what is predicted by the candidate elements.
The observations modeled as highly probable misses will hardly contribute to the gradients of the candidate elements.

In practice, the optimization is carried out in alternating stages.
In odd numbered stages, only the resolution parameters are tuned at a higher learning rate;
in even numbered stages, both the resolution and orbital elements are adjusted together at a slower learning rate.
There are some additional subtleties where the actual optimization function during the training of the mixture
parameters has a term to encourage the model to shrink the resolution and threshold parameters.
These will be discussed at greater length below.

As much as possible, I have sought to validate individual components of these calculations in isolation.
My numerical integration of the planets is validated against results from NASA JPL (Jet Propulsion Library)
using the superb Horizons system.
\footnote{
\href{https://ssd.jpl.nasa.gov/horizons.cgi}{NASA Horizons} \\
I cannot say enough good things about Horizons.
If you want an external ``gold standard'' of where an object in the solar system was or will be 
and a friendly user interface, Horizons is an excelent resource.}
I separately validated the numerical integration of the first 16 asteroids against positions and velocities obtained from Horizons.

The notion of a direction in space from an observer on Earth is typically reported in telescopic data using a right ascension and declination.
While these are convenient and standard for reporting observed data, they are not well suited to the approach taken here.
All directions are represented internally in this project as a unit vector $\uvec = (u_x, u_y, u_z)$ in the barycentric mean ecliptic (BME) frame.
These calculations were validated in isolation by querying the Horizons system for both the positions of and directions to known asteroids.
It is vital that this calculation takes into account the finite speed of light.
Treating light travel as instantaneous leads to errors that are catastrophically large in this context, on scales in the arc minutes rather than arc seconds.

The end to end calculation of a direction from orbital elements was verified as follows.
I integrated the trajectories of all the known asteroids using a collection of orbital elements downloaded from JPL.
I then computed the nearest asteroid number to each ZTF asteroid, and the distance between the predicted direction and observed direction.
I reviewed the statistical distribution of these distances.
I observed that out of approximately 5.69 million observations from the ZTF dataset,
3.75 million (65.71\%) fall within 2.0 arc seconds of the predicted directions of known asteroids.
I took this as overwhelming evidence that these calculations were accurate.

To put this degree of precision in context, 1.0 arc second is a back of the envelope estimate of the 
precision with which a modern telescope can determine direection of an observation under ideal observational conditions.
\footnote{Discussion with Pavlos Protopapas}
If you were to use an approximation that observations were made at Earth's geocenter 
(i.e. you did not account for location of the observatorory on Earth's surface) 
you would already be making errors on the order of 3 arc seconds.
If you were to perform your calculations using the Sun's location as your coordinate origin rather than the Solar System barycenter,
you would make errors around 0.6 arc seconds.
I know because I made both of these errors in earlier iterations before squeezing them out!

I tested the capabilities of the search process with an increasingly demanding set of search tasks.
The first three search tasks involved recovering the elements of known asteroids.
I took a batch of 64 asteroids that appeared most frequently in the ZTF data set.
These asteroids were represented between 148 and 194 times in the data, 
where hits here are counted at a threshold of 2.0 arc seconds as before.
Here is a summary of the tests I ran:
\begin{itemize}
\item Initialize search with correct orbital elements, but resolution $R = 0.5 \degree$ and threshold $\tau = 2.0 \degree$.
All 64 elements were recovered to a resolution of 3.0 arc seconds and 4.6E-6 AU, matching on 162 hits.
\item Initialize search with small perturbation applied to orbital elements; $a$ by $1.0\%$, $e$ by $0.25\%$, $i$ by $0.05 \degree$,
remaining angles $f$, $\Omega$ and $\omega$ by $0.25 \degree$.
42 of 64 elements were recovered to 18 arc seconds and 2.6E-4 AU, matching on 118 hits
\item Initialize search with large perturbation applied to orbital elements; $a$ by $5.0\%$, $e$ by $1.0\%$, $i$ by $0.25 \degree$,
remaining angles $f$, $\Omega$ and $\omega$ by $1.0 \degree$.
12 of 64 elements were recoverd to 32 arc seconds and 4.5E-4 AU, matching on 98 hits.
In some cases, a different (but correct) set of orbital elements was obtained;
the perturbation was so large the search found a different asteroid.
\item Initialize a search with \textbf{randomly initialized} orbital elements.
Search against the subset of ZTF observations within $2.0$ arc seconds of a known asteroid.
This search converged on one set of orbital elements matching a real asteroid on the first batch of 64 random candidates.
\end{itemize}
The last last test was significantly more demanding in that it did not rely on known orbital elements.

The work encompassed in the first three tests above can be seen as a way to independently validate a subset of the known asteroid catalogue.
It can efficiently associate a large number of telescope observations with known asteroids,
which could in turn be used to further investigate those asteroids.
Analysis might include refining their estimated orbital elements, 
fitting the $H-G$ model of brightness (magnitude), or identifying some of them for further investigation if they meet criteria of interest,
e.g. orbits that will approach near to Earth in the future.

The main thrust of this work, however, is not on refining the existing asteroid catalogue, it is finding new asteroids.
The final search I ran was against the subset of ZTF observations that did not match any of the known asteroids.
Random initializations for orbital elements were tried.
Most of these initializations fail to converge on elements with enough hits to match real asteroids in the data,
but a small number do successfully converge.
I have identified 9 candidate elements for potentially new asteroids with 8 or more hits.
I have verified that none of the orbital elements modeled for these asteroids 
are obvious matches in the known asteroid catalogue, though some are arguably close.
I have also done an ad-hoc review of the ZTF records to ensure that they are plausibly belonging to the same object.
I believe that at least some of these candidate elements may represent new and unkown asteroids, and plan to submit them to the 
\href{https://www.minorplanetcenter.net/iau/mpc.html}{Minor Planet Center} for possible classification.

The ultimate goal of this project is not to simply perform a one time search of a dataset to identify some new asteroids.
The goal is rather to create a tool that will be of enduring use to astronomy community for solving the problem of 
searching for new asteroids given large volumes of telescopic data.
To that end, I plan to consult with Matt Holman and his colleagues at the Minor Planet Center to see what refinements and improvements
would be required to upgrade this from a tool I can use to one that is of wider use to the astronomy community.

%%% Local Variables:
%%% mode: latex
%%% TeX-master: t
%%% End:


% \chapter{Hook\footnote{Co-authored with my advisor}}\label{ch:1}
\chapter{Integrating the Solar System}\label{ch:1}
\section{Introduction}
\label{section_intro}
The calculation of planetary orbits is arguably the canonical problem in mathematical physics.
Isaac Newton invented differential calculatus while working on this problem, and used his theory of gravitation to solve it.
In the important special case that one body in the system is a dominant central mass,
and all other bodies are viewed as massless ``test particles'', then a simple closed form solution is possible.
This formulation of the gravitational problem is often called the \href{https://en.wikipedia.org/wiki/Kepler_problem}{Kepler Problem},
named after \href{https://en.wikipedia.org/wiki/Johannes_Kepler}{Johannes Kepler}.
Kepler first studied this problem and published his famous \href{https://en.wikipedia.org/wiki/Kepler\%27s_laws_of_planetary_motion}{three laws of planetary motion},
the first of which states that the planets move in elliptical orbits with the sun at one focus.
This is a surprisingly good approximation for the evolution of the solar system, and the basis for the efficient linearized search over orbital elements developed in this thesis.

The two body approximation is not, however, sufficiently accurate for a high precision model of the past and future positions of the known bodies in the solar system.
While the mass of the sun is much larger than that of the heaviest planet, Jupiter, the planets are sufficiently massive
(and often closer to each other and other bodies of interest) that gravity due to their mass must also be accounted for.
The modern approach to determining orbits in the solar system is to use numerical integrators of the differential equations of motion.

\section{The \tty{REBOUND} Library for Gravitational Integration}
\label{section_rebound}
\tty{REBOUND} is an open source library for numerically integrating objects under the influence of gravity.
It is available on \href{https://github.com/hannorein/rebound}{GitHub}.
It is a first rate piece of software and I would like to thank Matt Holman and Matt Payne for recommending it to me last year.
At the end of Applied Math 225, I wrote a research paper in which I learned to use this library, 
extensively tested it on the solar system, and used it to simulate the near approach of the asteroid Apophis to Earth that will take place in 2029.
In this project, I use \tty{REBOUND} as the ``gold standard'' of numerical integration.
Because of its important role, I describe below how the \tty{IAS15} integrator I selected works.
\footnote{\tty{REBOUND} provides a front end to use multiple integrators. In this project, I make exclusive use of the default \tty{IAS15} integrator.}

The \tty{IAS15} integrator, presented in a 2014 paper by Rein and Spiegel, is a an impressive achievement.
It a fast, adaptive, 15th order integrator for the N-body problem that is (amazingly!) 
accurate to machine precision over a billion orbits.  
The explanation is remarkably simple in comparison to what this algorithm can do.  
Rein and Spiegel start by writing the equation of motion in the form 
$$y'' = F[y', y, t]$$
Here $y$ is the position of a particle; $y'$ and $y''$ are its velocity and acceleration;
and $F$ is a function with the force acting on it over its mass.
In the case of gravitational forces, the only dependence of $F$ is on $y$; 
but one of the major advantages of this framework is its flexibility to support other forces,
including non-conservative forces that may depend on velocity.
Two practical examples are drag forces and radiation pressure.

This expression for $y''$ is expanded to 7th order in $t$, 
$$y''[t] \approx y_0'' + a_0t + a_1t^2 + \cdots +a_6 t^7$$
They next change variables to dimensionless units $h = t / dt$ and coefficients $b_k = a_k dt^{k+1}$:
$$y''[t] \approx y_0'' + b_0h + b_1h^2 + \cdots + b_6 h^7$$
The coefficients $h_i$ represent relative sample points in the interval $[0, 1]$ that subdivide a time step.
Rein and Spiegel call them substeps.  
The formula is rearranged in terms of new coefficients $g_k$ with the property that $g_k$ depends
only on force evaluations at substeps $h_i$ for $i \le k$.
$$y''[t] \approx y_0'' + g_1h + g_2h(h-h_1) + g_3h(h-h_1)(h-h_2) + \cdots + g_8 h (h-h_1) \cdots (h-h_7)$$
Taking the first two $g_i$ as examples and using the notation $y_n'' = y''[h_n]$,
$$g_1 = \frac{y_1'' - y_0''}{h_1} \quad\quad  g_2 = \frac{y_2'' - y_0'' -g_1h_2}{h_2(h_2-h_1)}$$
This idea has a similar feeling to the Jacobi coordinates: a change of coordinates
with a dependency structure to allow sequential computations.

Using the $b_k$ coefficients, it is possible to write polynomial expressions for $y'[h]$ and $y''[h]$:
\begin{align*}
y'[h] &\approx y_0' + h dt \left(y_0'' + \frac{h}{2}\left(b_0 + \frac{2h}{3}\left(b_1 + \frac{}{} \cdots \right)\right) \right) \\
y[h] &\approx y_0 + y_0' h dt + \frac{h^2dt^2}{2}\left(y_0'' + \frac{h}{3}\left(b_0 + \frac{h}{2}\left(b_1 + \frac{}{} \cdots \right)\right) \right)
\end{align*}

The next idea is to use \href{http://mathworld.wolfram.com/RadauQuadrature.html}{Gauss-Radau quadrature}
to approximate this integral with extremely high precision.  
Gauss-Radau quadrature is similar to standard Gauss quadrature for evaluating numerical integrals, 
but the first sample point is at the start of the integration window at $h=0$.
This is a strategic choice here because we already know $y'$ and $y''$ at $h=0$ from the previous time step.
This setup now reduces calculation of a time step to finding good estimate of the coefficients $b_k$.
Computing the $b_k$ requires the forces during the time step at the sample points $h_n$,
which in turn provide estimates for the $g_k$, and then feed back to a new estimate of $b_k$.

This is an implicit system that Rein and Spiegel solve efficiently using what they call a predictor-corrector scheme.
At the cold start, they set all the $b_k=0$, corresponding to constant acceleration over the time step.
This leads to improved estimates of the forces at the substeps, and an improved estimates for the path on the step.
This process is iterated until the positions and velocities have converged to machine precision.
The first two time steps are solved from the cold start this way.  

Afterwards, a much more efficient initial guess is made.  
They keep track of the change between the initial prediction of $b_k$ and its value after convergence,
calling this correction $e_k$.  At each step, the initial guess is $b_k$ at the last step plus $e_k$.
An adaptive criterion is used to test whether the predictor-corrector loop has converged.
The error is estimated as 
$$\widetilde{\delta b_6} = \frac{\max_i |\delta b_{6,i}|}{\max_i |y_i''|}$$
The index $i$ runs over all 3 components of each particle.
The loop terminates when $ \widetilde{\delta b_6} < \epsilon_{\delta b}$; they choose $\epsilon_{\delta b} = 10^{-16}$.
It turns out that the $b_k$ behave well enough for practical problems that this procedure will
typically converge in just 2 iterations!

The stepsize is controlled adaptively with an analogous procedure.
The tolerance is set with a dimensionless parameter $\epsilon_b$,
which they set to $10^{-9}$.
As long as the step size $dt$ is ``reasonable'' in the sense that it can capture
the physical phenomena in question, the error in $y''$ will be bounded by the last term
evaluated at $h=1$, i.e. the error will be bounded by $b_6$.
The relative error in acceleration $\widetilde{b_6} = b_6 / y''$ is estimated as
$$ \widetilde{b_6} = \frac{\max_i |b_{6,i}|}{\max_i |y_i''|}$$
These are similar to the error bounds for convergence of the predictor-corrector loop,
but involve the magnitude of $b_6$ rather than its change $\delta b_6$.
An immediate corollary is that changing the time step by a factor $f$ will change $b_6$
by a factor of $f^7$.

An integration step is computed with a trial step size $dt_{\text{trial}}$.
At the end of the calculation, we compute the error estimate $\widetilde{b_6}$.
If it is below the error tolerance $\epsilon_b$, the time step is accepted.
Otherwise, it is rejected and a new attempt is made with a smaller time step.
Once a time step is accepted, the next  time step is tuned adaptively according to
$dt_{\text{required}} = dt_{\text{trial}} \cdot \left( \epsilon_b / \widetilde{b_6}\right)^{1/7}$
Please note that while the relative error in $y''$ may be of order 7, 
the use of a 15th order integrator implies that 
shrinking the time steps by a factor $\alpha$ will improve the error by a factor of $\alpha^{16}$.

% Hein and Spiegel include an interesting discussion of the different sources of error in N-body integrators.
% They explore the familiar errors arising from the numerical scheme, 
% but also explore both random and biased numerical errors.
% They give a complete error decomposition
% $$E_{\text{tot}} = E_{\text{floor}} + E_{\text{rand}} + E_{\text{bias}} + E_{\text{scheme}}$$
% $E_{\text{floor}}$ is the baseline numerical error that is unavoidable when we try to represent
% real numbers to limited machine precision.
% $E_{\text{rand}}$ is the familiar errors due to accumulated numerical round-off, 
% provided they are distributed randomly (i.e. unbiased).  
% $E_{\text{bias}}$ is an accumulated effect of numerical round-off that has a bias.

% This is a subtle point best illustrated by an example.
% Suppose you have a floating point representation of a 2x2 rotation matrix.
% For a given angle $\phi$, the floating point sum of $\cos^2\phi + \sin^2\phi$
% is likely not to be 1.0 to full machine precision.  
% If it is even a tiny bit less than 1, then repeated application of this rotation matrix
% over many time steps will gradually lead to a contraction.
% Rein and Spiegel devote substantial effort to minimizing rounding errors, 
% particularly by using \href{https://en.wikipedia.org/wiki/Kahan_summation_algorithm}{compensated sums}.
% This simple idea adds only slightly to the run time but can lead to significant accuracy improvements,
% especially over long term simulation, e.g. on the order of one billion orbital periods.

% Rein and Spiegel test the phase accuracy of \tty{IAS15} by running integrating the 
% outer solar system forward in time for 50 orbits with a fixed time step,
% then backwards for 50 orbits with the same time step.
% The known answer is that the phases should be the same, allowing for sharp accuracy measurements.
% This test shows that \tty{IAS15} is more accurate than the comparisons including WH at preserving phases.

% They also introduce a criterion of optimality and demonstrate that \tty{IAS15} integrator satisfies it.
% A result called Brouwer's Law status that in the presence of round-off, 
% a cumulative sum will have have an error that scales as $n^{1/2}$, i.e. it is a random walk of $n$ steps.
% Angular type variables including the orbital phase grow as $n^{3/2}$.
% The authors tested \tty{IAS15} and other integrators on a long term integration of the outer solar system
% (Jupiter, Saturn, Uranus, Neptune) for one billion Jupiter orbits, approximately 12 billion years.
% The energy errors satisfied Brouwer's law, and grew more slowly than those made by the other schemes.
% One surprising conclusions is that the non-symplectic \tty{IAS15} integrator is ``more symplectic''
% (in the sense of having smaller energy errors) than the symplectic integrators!
% This is a completely non-obvious results, and shows the importance of analyzing and understanding
% all the sources of error in a calculation.

\section{A Brief Review of the Keplerian Orbital Elements}
\label{section_orbital_elements}
In his work on the two body problem and the orbits of the planets, Kepler defined six 
\href{https://en.wikipedia.org/wiki/Orbital_elements}{orbital elements}
that are still in use today.
A set of orbital elements pertains to a body as of a particular instant in time, which is typically referred to as the ``epoch'' in this context.
The data sources I've seen all describe the time as a floating point number in the \href{https://en.wikipedia.org/wiki/Julian_day}{Modified Julian Day} (mjd) format.
In particular, I obtained orbital elements for all the known asteroids from \href{https://ssd.jpl.nasa.gov/?sb_elem}{JPL small body orbital elements}
as of MJD 58600, corresponding to 27-Apr-2019 on the Gregorian calendar.\\
Here is a brief review of the definitions of these orbital elements

\begin{figure}
\begin{center}
\includegraphics[width=0.40\textwidth]{orbital-elements-wikipedia.png}
\caption{Definition of the traditional Keplerian
\href{https://en.wikipedia.org/wiki/Orbital_elements}{orbital elements}
orbital elements, courtesy of Wikipedia.\\
Two parameters define the shape and size of the ellipse;
two define the orientation of the orbital plane; 
and the last two orient the ellipse in its plane and the phase of the body on its ellipse.}
\end{center}
\end{figure}

\begin{samepage}
\begin{itemize}
\item $a$, the semi-major axis; named \tty{a} in JPL and REBOUND
\item $e$, the eccentricity; named \tty{e} in both systems
\item $i$, the inclination; named \tty{i} in JPL and \tty{inc} in REBOUND
\item $\Omega$, the longitude of the ascending node; named \tty{node} in JPL and \tty{Omega} in REBOUND
\item $\omega$, the argument of perihelion; named \tty{peri} in JPL and \tty{omega} in REBOUND
\item $f$, the true anomaly; named \tty{f} in REBOUND; not quoted directly by JPL
\item $M$, the mean anomaly; named \tty{M} in both systems
\item mjd, the epoch as a Modified Julian Date
\end{itemize}
\end{samepage}
Distances are in A.U. in both JPL and REBOUND.  \\
Angles are quoted in degrees in JPL and in radians in REBOUND.

These orbital elements have stood the test of time because they are useful and intuitive.
They are ideal for computations, both theoretical and numerical, because in the case of the two body problem five of the six orbital elements remain constant.
The careful reader will note that there are 8 entries in the table above, but I've described elements as coming six at a time.
The epoch is considered to be the ``seventh element'' because in the Kepler two body problem, we can describe one body at different times, but it will have the same orbit.
This point of view extends to the N-body problem, which is fully reversible; the same system can be described at at different moments in time.
In practice, the orbital elements are often used to describe the initial conditions of all the bodies for an integration.
The problem is then integrated numerically, possibly both forwards and backwards.
Orbital elements can be reported for any body of interest.

A body orbiting the sun has six degrees of freedom.  
In Cartesian coordinates, there are three for the position and three for the velocity.
In orbital elements, the first five are almost always $(a, e, i, \Omega, \omega)$.
These five will remain constant for a body moving in the Kepler two body problem.

There is some variation in the choice of the sixth element, because different representations have different pros and cons.
The true anomaly $f$ is most convenient for transforming back and forth between orbital elements and Cartesian space.
The mean anomaly $M$ is most convenient for studying the time evolution of the system, because it changes linearly with time in the Kepler two body problem.
The mean anomaly and true anomaly are related by the famous 
\href{https://en.wikipedia.org/wiki/Kepler\%27s_equation}{Kepler's Equation}
This relates the mean anomaly $M$ to the eccentric anomaly $E$.
The \href{https://en.wikipedia.org/wiki/Eccentric_anomaly}{eccentric anomaly} is yet another angle describing a body in orbit.
\begin{figure}
\begin{center}
\includegraphics[width=0.40\textwidth]{orbital-anomalies.png}
\caption{Three Orbital Anomalies: Eccentric, Mean and True}
\end{center}
\end{figure}

\begin{align*}
\tan \left(\frac{f}{2} \right) &= \sqrt{\frac{1+e}{1-e}} \tan \left( \frac{E}{2} \right) \\
M &= E - e \sin(E)
\end{align*}
 
The linear evoluation of the mean anomaly, along with Kepler's equation, allows us to efficiently compute orbits for the Kepler two body problem.
The relationship between the eccentric anomaly $E$ and true anomaly $f$ is a one to one function that can be evaluated fast on a computer.
The mapping from eccentric anomaly $E$ to mean anomaly $M$ is also fast.
The inverse mapping from $M$ to $E$ does not have a known analytical form.
But it can be evaluated raplidly using Newton's Method with a reasonable initial guess.
This is the method that I use to compute the orbits under the Kepler approximation. 

\section{Numerical Integration of the Planets and Asteroids}
\label{section_numerical_integration}

I have described above a library \tty{REBOUND} that can efficiently integrate the solar system,
and a data source Horizons that can be used to obtain accurate initial conditions for solar bodies.
In principle integrating the solar system is a straightfoward exercise.
In practice, there are quite a few details that need to be worked out before you can obtain reliably correct answers.
You need to carefully specify the bodies you submit to Horizons.
Horizons has separate identifiers for e.g. the barycenter of the Earth-Moon system, the Earth, and the Moon.

The module \tty{horizons.py} contains functions used to query the Horizons AP.
It also maintains a local cache with the results of prior queries; 
this yields significant savings in time because a typical horizons query using the Horizons API in \tty{REBOUND} takes about one second.
The main function in this module is \tty{make\_sim\_horizons}.
Given a list of object names and an epoch, it queries Horizons for their positions and velocities as of that date.
It uses this data to instantiate a \tty{REBOUND Simulation} object. \\
The module \tty{rebound\_utils.py} contains functions used to work with \tty{REBOUND} simulations.
It includes functions to build a simulation (\tty{make\_sim}).
This will seek to load a saved simulation on disk if it is available, otherwise it will query Horizons for the required initial conditions.
The function \tty{make\_archive} builds a \tty{REBOUND SimulationArchive}.
As the name suggets, a \tty{SimulationArchive} is a collection of simulation snapshots that  have been integrated.
This function also saves the integrated positions of the planets and test bodies as plain old \tty{Numpy} arrays for use in downstream computations.

The module \tty{planets.py} performs the numerical integration of the planets.
To be more precise, it will integrate different collections of massive bodies in the solar system
\begin{itemize}
\item \textbf{Planets}: The Sun; The Earth and Moon as separate bodies; and the barycenters of the other seven IAU planets 
Mercury, Venus, Mars, Jupiter, Saturn, Uranus, and Neptune (10 objects)
\item \textbf{Moons}: The 8 IAU planets, plus the following significant moons and Pluto (31 objects): \\
Jupiter: Io, Europa, Ganymede, Callisto \\
Saturn: Mimas, Enceladus, Tethus, Dione, Rhea, Titan, Iapetus, Phoebe \\
Uranus: Ariel, Umbriel, Titania, Oberon, Miranda \\
Neptune: Triton, Proteus \\
Pluto: Charon 
\item \textbf{Dwarfs}: All objects in the solar system with a mass at least $1E-10$ Solar masses (31 objects): \\
Planets: Earth, Moon, and barycenters of other seven planets \\
Above 1E-9: Pluto Barycenter, Eris, Makemake, Haumea \\
Above 1E-10: 2007 OR10, Quaoar, Hygiea, Ceres, Orcus, Salacia, Varuna, Varda, Vesta, Pallas
\item \textbf{All}: All objects in the solar system with a mass at least $1E-10$ Solar masses (45 objects): \\
All 8 planets (not barycenters) \\
All the heavy moons above \\
All the dwarf planets above
\end{itemize}

Each configuration above was integrated for a 40 year period spanning 2000-01-01 to 2030-12-31 and a time step of 16 days.
I tested the integration by comparing the predicted positions of the 8 planets to the position quoted by Horizons at a series of test dates .
The test dates are at 1 year intervals over the full 40 year span that is simulated.
The best results were obtained by integrating smallest collection: Earth, Moon, and the barycenters of the other 7 planets.
I was a bit surprised at this result and expected to do slightly better as the collection of objects became larger.
Position errors are reported in AUs, with the root mean square (RMS) error over the 40 annual dates.
I also compute an angle error by comparing the instantaneous direction from each planet to Earth geocenter in the BME frame.
I reported errors on this basis because on this problem, everything is done in terms of directions so precision eventually
comes down to a tolerance in arc seconds.

\begin{table}
\begin{centering}
\begin{tabular}{ | c | c | c |}
\hline
\multicolumn{1}{|p{2cm}|}{\centering Object \\ Collection } & 
\multicolumn{1}{|p{2cm}|}{\centering Position \\ Error} & 
\multicolumn{1}{|p{2cm}|}{\centering Angle \\ Error} \\
\hline
Planets & 5.38E-6 & 0.79 \\
Moons & 1.35E-5 & 0.81 \\
Dwarfs & 5.38E-6 & 0.79\\
All & 1.35E-5 & 0.81 \\
\hline
\end{tabular}
\caption{Root Mean Square Error in Integration of Planets vs. Horizons\\
Position Error: RMS error of 8 planets in AU.\\
Angle Error: RMS error in direction from planet to Earth geocenter, in Arc Seconds}
\end{centering}
\end{table}

While it might at first seem surprising that the results are worse for the more complex integrations including the moons,
it's important to realize that this problem is intrinsically more difficult.
Simulating the evolution of the barycenter of e.g. the Jupiter system is significantly easier than keeping track of the heavy moons and integrating them separately.
Overall these results are excellent; over a span of 20 years in either direction, integrations are accurate on the order of $10^{-6}$ AU.
The angular precision on the order of $\sim 0.8$ arc seconds is also excellent for such a long time span and well within the tolerance of this application.

After reviewing these results, I decided that the optimal strategy for the asteroid search problem was to treat the heavy bodies 
in the solar system as the smallest collection, shown on row 1.
It is necessary to model the position of the Earth and Moon separately rather than the Earth-Moon barycenter, 
since our observatories are on the planet, not relative to the planetary system barycenter.
However, the role of the other planets is only as a gravitational attractor that deflects the orbit of the Earth and the Asteroids.
Speed is important in this application, so the smallest and fastest collection was the clear choice.

The second test of the integration of the planets was a ``soup to nuts'' test with the integration of the planets, plus ten test asteroids.
I selected as the test asteroids the first 10 IAU numbered asteroids: Ceres, Pallas, Juno, Vesta, Iris, Hygiea, Egeria, Eunomia, Psyche, Fortuna.
This test does not yet exercise the part of the code that insantiates asteroid orbits based on the bulk orbital elements files; that comes later.
The asteroids here are initialized the same way as the planets, by querying the Horizons API in \tty{REBOUND}.
(This method would not scale up to integrating all the asteroids though, because it is far too slow at about 1 second per asteroid.)
The test protocol here was the same as for the planets.
I compared the positions of these asteroids in the barycentric mean ecliptic frame predicted by my integartion at annual dates to the positions quoted by JPL.
I also compared the instananeous angle from Earth geocenter to the asteroid. \\
Below are two charts summarizing the results.
\begin{figure}
\begin{center}
\includegraphics[width=1.0\textwidth]{sim_pos_error_comp.png}
\includegraphics[width=1.0\textwidth]{sim_ang_error_comp.png}
\caption{Position and Angle Error of 10 Test Asteroids. \\
My integration is compared to positions extracted from Horizons at 40 dates from from 2000 to 2040.}
\end{center}
\end{figure}
If we focus on a plausible window of $\pm 5$ years around 2020, we can see that the selected planets integration is extremely accurate.
Angular errors 5 years out are on the order of $0.5$ arc seconds.

\section{Efficient Integration of All Known Asteroids}
\label{section_integrate_known_asteroids}

\section{Integration of Kepler Two Body Problem in Tensorflow}
\label{section_kepler_two_body_tensorflow}


\chapter{Predicting Directions from Positions}\label{ch:2}
\section{Introduction}
\label{section_intro}
In this chapter we will relate the integrated path of an asteroid to its appearance to an observer on Earth.
We will review the equatorial coordinate system, which describes the location of an object in the sky 
with the parameters right ascension (RA) and declination (DEC).
We will develop the calculations to transform between a direction from earth represented as a pair (RA, DEC)
to a direction represented as a unit vector $\uvec = (u_x, u_y, y_z)$ in the barycentric mean ecliptic frame.
We will explore the ZTF dataset of telescopic observation from the Palomar observatory in California.
We will compute the direction $\uvec$ from which an observer at Palomar would have seen an object at a given observation time
as a function of its predicted position $\qvec$ and velocity $\vvec$ at that observation time.
This calculation will account for the time required for light to reach the observatory (``light time'')
and for the location of the observatory on the surface of the Earth as distinct from geocenter (``topos adjustment'').
We will run this calculation on all the known asteroids, computing the direction they would have appeared in the sky if they had been visible at Palomar.
We will use this calculation to associate each ZTF obervation with the nearest asteroid to it in the sky.

Finally, we will study the statistical distribution of angular distance between the ZTF observations and the nearest asteroid.
We will show that 65.71\% of these observations fall within 2.0 arc seconds of the direction predicted for one of the 733,489 catalogued asteroids.
We will compare this to the theoretical distribution of angular distances to the nearest asteroid 
if the predicted directions were distributed uniformly on the sphere.
We will show that such a high preponderance of ``hits'' is wildly unlikely and conclude that the asteroid catalogue 
has correct orbital elements for the objects being detected, 
and that the combined tolerance of the instruments and this calculation apparatus is on the order of 2.0 arc seconds.

\section{A Brief Review of Right Ascension (RA) and Declination (DEC)}
\label{section_ra_dec}
How can we describe the direction of an object we see in the sky?
It is a question that dates back to the first astronomers in ancient times.
The simplest and most intuitive coordinate system is the \href{https://en.wikipedia.org/wiki/Horizontal_coordinate_system}{topocentric coordinate system},
which uses the local horizon of an observer on the surface of the Earth as the fundamental plane.
In this coordinate system, an object in the sky is described in terms of an altitude (sometimes called an elevation) and an azimuth.
The altitude is how many degrees the object is above the visible horizon, so it will be between $0 \degree$ and $90 \degree$.
The topocentric coordinate system is intuitive and easy to use for an observer standing on the surface of the Earth and looking at the sky.
If I wanted to do some amateur astronomy with my kids and know where to point an inexpensive telescope, these are the coordinates I would want.
\begin{figure}[hbt!]
\begin{center}
\includegraphics[width=0.6\textwidth]{topocentric_coordinates.png}
\caption{The topographic coordinate system, courtesy of \href{https://en.wikipedia.org/wiki/Horizontal_coordinate_system}{Wikipedia}}
\end{center}
\end{figure}

But the topcentric coordinate system is poorly suited to sharing observational data between astronomers.
It is a different reference frame depending on where you are located on the Earth and the time in the evening.
For this reason, astronomers dating back to ancient times developed the \href{https://en.wikipedia.org/wiki/Equatorial_coordinate_system}{equatorial coordinate system}.
This coordinate system draws an imaginary sphere around the Earth.
The $x$ axis of this coordinate frame points from the Sun to the center of the Earth at the Vernal Equinox of a specific date 
(typically \href{https://en.wikipedia.org/wiki/Epoch_(astronomy)}{J2000.0} these days).
The $y$ axis is 90 degrees to the East and the $z$ axis points to the North pole.
This is much easier to understand with a picture than with words:
\begin{figure}[hbt!]
\begin{center}
\includegraphics[width=0.8\textwidth]{celestial_sphere.png}
\caption{The celestial sphere, courtesy of \href{http://coolcosmos.ipac.caltech.edu/cosmic_classroom/cosmic_reference/coordsys.html}{cool cosmos}.}
\end{center}
\end{figure}

The celestial sphere was originally conceived as having its $z$ axis pointing along the axis of the Earth's rotation, i.e. from the South Pole to the North Pole.
This is not a good definition though for the purposes of modern astronomy, 
because the Earth's rotational axis is not a fixed direction in the barycentric mean ecliptic frame.
The Earth's rotational axis changes over time due to two separate effects: 
\href{https://en.wikipedia.org/wiki/Axial_precession}{precession} and \href{https://en.wikipedia.org/wiki/Astronomical_nutation}{nutation}.
Precession is a low frequency, predictable effect that in analagous to wobbling movement of a top's rotational axis.
It is typically covered in a first semester undergraduate physics course on mechanics.
The precession of the Earth's axis is caused by mainly by the gravitational influence of the Sun and Moon, and has a period of approximately 25,772 years.
Nutation is a high frequency effect, also caused mainly by the gravitational foreces of the Sun and Moon.
In fact, the only difference bewteen the two effects is in the way we model and understand them.
Precession isolates the mean perturbation over time; it's a slow, steady drift that is easy to predict.
Nutation describes the much smaller, high frequency wobbles (on the order of tens of arc seconds) that are the residual after precession is accounted for.

Because of the instability of the Earth's rotational axis, the definition of the directions were updated to refer to the mean ecliptic as of a date,
rather than Earth's rotational axis when the measurement is taken.
Detailed calculations of the precession and nutation can then be used to adjust measurements taken on different dates.
Ineed, the \tty{astropy} astronomy package has the capability to perform these calculations.
The diagram of the equataorial coordinate system clarifies this more modern definition.  
The vernal equinox is the moment when the Earth's orbit crosses the ecliptic.

This definition of a celestial coordinate system has in turn been superseded by a still more accurate system:
the Interational Celestial Reference System, \href{https://en.wikipedia.org/wiki/International_Celestial_Reference_System}{ICRS}.
The ICRS is based on an elegant idea.
The goal of a celestial reference frame is to provide directions that are as near to fixed as possible in the reference frame of the Solar System.
Astronomers realized that since objects that are very far away from the Milky Way have orientations that are effectively fixed,
the most precise way to define directions was based on large quantities of astronomical observations of these objects.
In particular, radioastronomy (observations of stars in the radio wave frequency) is used.
This is the basis of the ICRS and the frame of reference it defines: the International Celestial Refrence Frame (ICRF).
The ICRF is a 3 dimensional coordinate system whose origin is the Solar System barycenter (center of mass).
The $X$, $Y$ and $Z$ axes are set by convention to line up very closely to the traditional definition of the J2000.0 frame.
The orientation of the coordinate axes is based on the measured positions of 212 extragalactic objects, mainly quasars.
These objects are so far away from Earth and the Milky Way galaxy that they are considered ``fixed points'' in space.
\footnote{\href{http://kejian1.cmatc.cn/vod/comet/oceans/naval_observatory/navmenu.php_tab_1_page_3.2.1_type_text.htm}{U.S. Naval Observatory - ICRS}}
Modern telescopes can achieve extremely accurate RA/Dec measurements by calibrating against the measured directions to these objects.
\begin{figure}[hbt!]
\begin{center}
\includegraphics[width=0.8\textwidth]{ICRS_USNO.jpg}
\caption{The International Celestial Reference Frame (ICRF), courtesy of 
\href{http://kejian1.cmatc.cn/vod/comet/oceans/naval_observatory/navmenu.php_tab_1_page_3.2.1_type_text.htm}{U.S. Naval Observatory}.}
\end{center}
\end{figure}

The main conclusion of this short section is that even an apparently simple idea, the direction from an observer on Earth to an object seen in the sky,
is quite subtle if you want to take reproducible measurements that are accurate on the order of arc seconds.
The ICRS is accurate on the order of a handful of milliarcseconds, 
meaning that uncertainty in the coordinate frame is not a meaningful contributor to errors for any calculations in this thesis.
Fortunately this is a mature and well studied problem in astronomy, and it is addressed well by the \tty{astropy} package.
Rather than trying to reinvent the wheel, I use the \tty{astropy} as much as possible in the next section
to relate RA/Dec measurements to directions $\uvec$ in the barycentric mean ecliptic frame.

\section{Mapping Between RA/Dec and Direction $\vec{u}$ in the Ecliptic Frame}
\label{section_ra_dec_to_dir}
In the previous section we have explored the ICRS, which provides the definition of the RA/Dec measurements quoted by astronomers.
In this section, I review the definition of the barycentric mean ecliptic frame that is used for all calculations in this thesis.
Then I explain the functions that are used to perform the conversions between RA/Dec and directions in the ecliptic frame.

The barycentric ecliptic frame is the natural choice for computations done in relation to integrations of the Solar System.
It defines the $xy$ plane to containing the mean ecliptic (ellipse of the Earth's orbit around the Sun).
We take the mean ecliptic because the ecliptic varies slowly over time; 
the rotation of the Earth around the Sun in any given year would not be contained \textit{exactly} in a plane,
but there is a mean plane that comes closest in the least square sense to hitting all the points.
The $z$ axis is the unique direction that is orthogonal to this plane.
Within the $xy$ plane, the $x$ axis is oriented from the Sun to Earth geocenter at the vernal equinox.
In particular, the $x$ axis is the direction from the Earth to the Sun at the vernal equinox as of the epoch, 
which is currently J2000.0 (JD 2451545.0, approximately January 1, 2000 12:00 UTC on the Gregorian calendar).

Here the vernal equinox is not defined with its ancient interpretation as the day when the day and night have equal lengths,
but by the modern astronomical definition as the intersection of two planes:
the mean ecliptic plane described above, and the mean equator as of that date.
The idea of the mean equator as of a date is that as described above, the Earth's equator undergoes small wobbles at high frequency due to nutation.
The mean equator as of a date disregards these wobbles, and instead considers only the trend of the equator, which changes slowly due to precession.
This definition has the advantages of being more precisely measurable and changing more slowly than the true equator, making it the standard.

\begin{figure}[hbt!]
\begin{center}
\includegraphics[width=0.8\textwidth]{heliocentric_ecliptic.png}
\caption{The Heliocentric Ecliptic Frame, courtesy of \href{https://en.wikipedia.org/wiki/Ecliptic_coordinate_system}{Wikipedia}\\
The Barycentric Ecliptic Frame is analogous, but the origin is the Solar System barycenter rather than the Sun.}
\end{center}
\end{figure}

The module \tty{ra\_dec.py} handles conversions between RA/Dec and the direction $\uvec$ in the barycentric mean ecliptic (BME) frame.
The two workhorse functions are named \tty{radedc2dir} and \tty{dir2radec}.
The hard work in getting all of this to work comes in understanding the ideas of the coordinate systems and developing tests.
Once you know what everything means and how to test that the implementation is right, it amounts to just a few lines of code.
The \tty{astropy SkyCoordinate} class is aware of both the ICRF and the BME frames.
In order to transform between the two coordinate frames, we need only instantiate a \tty{SkyCoordinate} of the required type, 
and ask \tty{astropy} to transform it for us.
It's so easy I will include a code snippet to give the flavor; this is from \tty{radec2dir}:
\begin{lstlisting}[style=CodeSnippet]
obs_icrs = astropy.SkyCoord(ra=ra, dec=dec, obstime=obstime, frame=ICRS)
obs_ecl = obs_icrs.transform_to(BarycentricMeanEcliptic)
u = obs_ecl.cartesian.xyz
\end{lstlisting}

How can we test that all of this is working? 
By comparing my calculations to those done by the JPL.
At first I struggled to reconcile my results.
I downloaded the computed position of Earth and Mars from JPL as well as the RA/Dec predicted by JPL for an imaginary observer at Earth geocenter.
When I compared the JPL results to my own, they did not match,
I eventually realized that the JPL calculation is accounting for light delay.
In the next section, I wil explain this calculation and demonstrate that my results are consistent with JPL.

\section{Computing a Direction $\vec{u}$ from Position $\vec{q}$ and Velocity $\vec{v}$}
\label{section_pos_vel_to_dir}
Suppose we have computed the position $\qearth$ and $\qast$ where we believe the Earth and an asteroid are at the same time $t$.
Can we compute the direction $\uvec$ from the Earth to the asteroid by simply subtracting the poitions and normalizing them, i.e. by
$$ \uvec \stackrel{?}{=} \frac{\qast - \qearth}{\norm{\qast - \qearth}}$$
The answer is no! This fails to account for the finite speed of light.
The photons arriving at the observatory at the observation time $t_1$ were not emitted at $t_1$; they were emitted in the past, at time $t_0$.
\begin{figure}[hbt!]
\begin{center}
\includegraphics[width=0.8\textwidth]{light_time.png}
\caption{Computing the Apparent Direction from Earth to an Asteroid, Accounting for the Speed of Light}
\end{center}
\end{figure}
It's straightforward to calculate a correction that is accurate to first order once we draw a picture.
It requires that we know both the predicted position and the predicted velocity of the asteroid at time $t_1$.
\begin{align*}
\mathbf{q}_{\mathrm{rel}} &= \qast - \qearth \\
T_{\mathrm{light}} &= \norm{ \mathbf{q}_{\mathrm{rel}} } / c \\
\Delta \qast &= \vvec_{\mathrm{ast}} \cdot T_{\mathrm{light}} \\
\mathbf{q}_{\mathrm{lt}} &= \mathbf{q}_{\mathrm{rel}} - \Delta \qast \\
\uvec &= \mathbf{q}_{\mathrm{lt}} / \norm{\mathbf{q}_{\mathrm{lt}}}
\end{align*}
Here $c$ is the speed of light as usual.

One slightly counterintuitive fact is that we don't need to know or account for the speed of the Earth, 
or compute the relative velocity of the asteroid to the Earth. 
Why is this?
We are performing all of our calculations in the barycentric mean ecliptic frame.
This is very nearly an inertial frame of reference, so physics behaves ``nicely'' and everything works as expected.
The only small departures from this frame being inertial are due to the orbit of the Solar System around the Milky Way
and the acceleration of the Milky Way in the entire universe.
If we used a different reference frame, e.g. the heliocentric frame, we would make errors due to the acceleration of this frame 
unless we accounted for them, which would be equivalent to the calculations shown here.
In the BME frame, we have determined the positions of both the Earth and the asteroid at the instant $t_1$ at which 
photons landed on the detector of the observatory.
To calculate the direction from which these photons arrived in the BME frame, we need only know where the asteroid was at the moment
$t_0$ when photons emitted from it would have arrived at $t_1$.
This calculation of an astrometric direction is implemented in the function \tty{astrometric\_dir}, also in \tty{ra\_dec.py}.
It takes as inputs \tty{q\_body}, \tty{v\_body} and \tty{q\_obs}, and returns the astrometric direction $\uvec$
from the observer to the body accounting for light time.

It's worth pointing out here that a more fully correct description would be given by solving
$$ \norm{\qearth(t_1) - \qast(t_0)} = c \cdot (t_1 - t_0) $$
This is the approach taken by the astronomy library \tty{SkyField}, which solves this equation iteratively.
Each iterative step matches the first order solution shown above.
For the purpose of this problem, the light time between an asteroid and the Earth will typically be on the order of 20 minutes or so.
(The speed of light in AU / minutes is 0.120, so 20 minutes of light travel covers 2.4 AU).
This is short enough time interval that approximating the motion of the asteroid as linear is highly accurate.

We can also use this approach to approach to make a back of the envelope estimate of the errors we might make if 
we disregarded the light time adjustment
Suppose an asteroid has $a=3$, so by Kepler's thrid law this body would have an orbital period of $3^{3/2} \approx 5.2$ years.
At a time when this asteroid is 3.0 AU from Earth, the light time would be 25 minutes, which workds out to 2.20E-4 of its orbital period.
Converting this into degrees is a multiplication by 360, and into arc seconds a further multiplication by 3600.
I estimate that ignoring light time could lead in this case to errors on the order of 285 arc seconds 
if the asteroid were moving perpendicular to its displacement to Earth.
While that's close enough to aim an amateur's optical telescope, it's a catastrophic error here.

There is one more correction we need to account for: the position of the observatory on the surface of the Earth, or ``topos''.
Our integration of the Solar System gives us $\qearth$, the position of the Earth's center of mass in the BME as a function of time.
But our observatory of course sits on the surface of the Earth, not at the center.
It would be too hot in the center, plus you couldn't see anything from there.
Fortunately this is another well studied problem that is handled well by mature software libraries.
I spent a lot of time trying to get a solution working based on the implementation in \tty{astropy},  but I simply could not get it to work.
Eventually I gave up and decided to use \tty{SkyField} instead.
The function \tty{calc\_topos} in \tty{ra\_dec.py} computes the topos adjustment at a named observatory site as of a vector of times.
Currently the only observatory site required is Palomar mountain in California.
The topos correction is a pair of corrections $\Delta \qvec_{\mathrm{topos}}$ and $\Delta \vvec_{\mathrm{topos}}$ such that
$$\qobs = \qearth + \Delta \qvec_{\mathrm{topos}}$$
The spline is generated with the following code snippet (slightly edited for brevity)
\begin{lstlisting}[style=CodeSnippet]
obsgeoloc = SkyField.EarthLocation.of_site(site_name))
longitude, latitude, height = obsgeoloc.geodetic
topos = SkyField.Topos(latitude=latitude, longitude=longitude, elevatiom=elevation)
dq_topos = topos.at(obstime_sf).ecliptic_position().au.T * au
\end{lstlisting}

The function \tty{qv2dir} combines the capabilities of \tty{astrometric\_dir} and \tty{calc\_topos}.
It takes as inputs \tty{q\_body}, \tty{v\_body}, \tty{q\_earth}, \tty{obstime\_mjd} and \tty{site\_name}.
It returns the astrometric direction from an observer at the named site on Earth, 
to a body with the victor position and velocity vectors aligned with the observation times.

The functions described above are based on \tty{numpy} arrays and run normally on the CPU.
I also created a custom Keras Layer in TensorFlow that performs these calculations called \tty{AsteroidDirection}.
At initialization, an \tty{AsteroidLayer} requires an array \tty{ts} of MJDs, the row lengths for each observation in the batch, and the site name.
The main \tty{call} method of this layer accepts seven inputs: the orbital elements including the epoch, namely
$(a, e, i, \Omega, \omega, f, t_0)$.
It returns the predicted astrometric direction $\uvec$ from the named observatory to an asteroid with these orbital elements.
This is the core layer that is used to predict directions to candidate orbital elements during the asteroid search.

We are now ready to review the demonstration that my calculations are consistent with JPL and SkyField.
You can follow this demonstration interactively in the Jupyter notebook 
\tty{03\_RA\_DEC.ipynb} in the \tty{jupyter} directory of the project repository.
I downloaded from Horizons a data set with with the positions and velocities of Earth and Mars.
These sets contained 29,317 rows spanning a 10 year period from 2010-01-01 to 2020-01-01 sampled at 3 hour intervals.
I also downloaded from Horizons what they refer to as ``observer'' calculations including a RA/Dec.
I specified an observer location at Palomar Mountain, which is a preset observatory in Horizons.
The SkyField calculations use a slick end to end capability to predict where an observer would see an object.
It uses a downloaded JPL ephemeris file, but is otherwise a self contained calculation that is independent from both JPL and my calculations.

I compared my calculations to two sources: JPL and SkyField, as well as comparing SkyField to JPL.
The angular distance between two directions on the unit circle can be calculated with a simple formula that I will review later.  
For very small distances, the angular difference in radians is equal to the Cartesian difference between the two direction
vectors $\uvec_1$ and $\uvec_2$ on the unit sphere in $\R^3$.

Here is a table summarizing the mean difference between Horizons, SkyField and my calculations:
\begin{table}
\begin{centering}
\begin{tabular}{|c | c|}
\hline
Sources & Difference \\
\hline
SKY vs. JPL & 1.598 \\
MSE vs. JPL & 1.604 \\
MSE vs. SKY & 0.027 \\
\hline
\end{tabular}
\caption{Mean differences between JPL, SkyField, and my calculations (MSE) in Arc Seconds\\
My results are essentially identical to SkyField.  Both SkyField and I disagree with JPL by 1.6 arc seconds.}
\end{centering}
\end{table}
My results are substantially identical to SkyField, to the miniscule tolerance of 0.027 arc seconds.
Both SkyField and I differ a tiny bit from JPL, to the tune of 1.6 arc seconds.

I did some further tests, this time comparing my predicted directions from Earth to the first 16 asteroids 
with the results of applying my \tty{radec2dir} function to the RA/Dec quoted by JPL,
with the directions I predicted using the integrated orbits.  In pseudocode, the test looks like this:
\begin{align*}
\uvec_{\mathrm{JPL}} &= \tty{radec2dir(RA\_JPL, DEC\_JPL)} \\
\uvec_{\mathrm{MSE}} &= \tty{qv2dir(q\_ast\_MSE, v\_ast\_MSE, q\_earth\_MSE, 'palomar')}
\end{align*}
This test is exercising both the integration and the angle conversion.
On 10 years of daily data for 16 asteroids, the root mean squared error is \textbf{\emph{0.873 arc seconds.}}

\section{Predicting the Apparent Magnitude of an Asteroid}
\label{section_mag}
\todo{FILL THIS IN}

\section{Conclusion}
\label{section_conclusion}
I have presented in this chapter the calculations required to determine the astromentric direction 
of an asteroid observed from Earth given its orbital elements.
This calculation takes into account the time for light to travel from the asteroid to the Earth
and the position of the observatory on the Earth's surface.
I have tested these calculations against two independent sources, NASA Horizons and the SkyField astronomy library.
I have demonstrated that my results are within 1.0 arc second of NASA and substantially identical to SkyField to within 0.01 arc seconds.
Finally, I have demonstrated a high speed TensorFlow implementation of these calculations in the \tty{AsteroidDirection} layer.
The TensorFlow model can predict astrometric directions of a set of candidate orbital elements 
along with the derivatives of these predicted directions with respect to the six orbital elements.


\chapter{Searching for Asteroids}\label{ch:3}
\section{Introduction}
\label{section_ztf_intro}
Zwicky Transient Facility (\href{https://www.ztf.caltech.edu/}{ZTF}) is a time-domain survey of the northern sky
that had first light at Palomar Observatory in 2017.  It is run by CalTech.
My advisor Pavlos suggested it as a data source for this project.
The ZTF dataset has two major advantages for searching for asteroids:
\begin{itemize}
\item ZTF gives a wide and fast survey of the key, covering over 3750 square degrees an hours to a depth of 20.5 mag
\item A machine learning pipeline has been developed to classify a subset of ZTF detections that are classified as probable asteroids
\end{itemize}
The data set I analyze here consists of all ZTF detections that were classified as asteroids.
Data on each detection include:
\begin{itemize}
\item \textbf{ObjectID} an identifier of the likely ojbect associated with this detection; multiple detections often share the same ObjectID
\item \textbf{CandidateID} a unique integer identifier of each detection
\item \textbf{MJD} The time of the detection as an MJD
\item \textbf{RA} The right ascension of the detection
\item \textbf{Dec} The declination of the detection
\item \textbf{mag} The apparent magnitude of the detection
\end{itemize}
Available data also includes a number of additional fields that were not used in the analysis.

\href{https://github.com/alercebroker}{ALeRCE} (Automatic Learning for the Rapid Classification of Events) is an astronomical data broker.
ALeRCE provides a convenient API to access the ZTF asteroid data, which can be installed with \tty{pip}.
I used ALeRCE on this project to download the ZTF asteroid data set.

\section{Exploratory Data Analysis of ZTF Asteroid Data}
\label{section_ztf_eda}
Before plowing into the search for new asteroids, I conducted an exploratory data analysis (EDA) of the ZTF asteroid dataset.
This can be followed interactively in the Jupyter notebook \tty{05\_ztf\_data.ipynb}.
I took a download of the data running through 26-Feb-2020.
The first detection is on 01-Jun2018.
The dataset contains 5.69 total detections.  
The volume of detections increases very significantly beginning in July 2019; 
for practical purposes the dataset consists of 8 months of detections spanning July 2019 through February 2020.

\begin{figure}[hbt!]
\begin{center}
\includegraphics[width=0.8\textwidth]{../figs/ztf/ztf_ast_per_month.png}
\caption{The topographic coordinate system, courtesy of \href{https://en.wikipedia.org/wiki/Horizontal_coordinate_system}{Wikipedia}}
\end{center}
\end{figure}


\section{Finding the Nearest Asteroid to Each ZTF Observation}
\label{section_ztf_nearest_ast}

\section{Analyzing the Distribution of Distance to the Nearest Asteroid}
\label{section_nearest_ast_distribution}
\section{Setup}

% \section{Conclusion}


%%%%%%%%%%%%%%%% BACK MATTER %%%%%%%%%%%%%%%%

% Put appendices, bibliography, and supplemental materials here

% The bibliography may be single spaced within each entry, but must be
% double-spaced between each entry. Most bibliography styles leave space between
% entries, so that shouldn't be a problem.
\begin{singlespacing}
% I like "References" better than "Bibliography"
\renewcommand{\bibname}{References}

% Any bibliohgraphy style that leaves space between entries is fine
% \bibliographystyle{acm}
% \bibliography{references}
\newpage
\begin{thebibliography}{9}

\bibitem{MOPS}
Denneau, Larry et al.\\
\textit{The Pan-STARRS Moving Object Processing System}\\
Astronomical Society of the Pacific.  April 2013.\\
Volume 125. Number 926\\
JSTOR: http://www.jstor.org/stable/10.1086/670337 \\
DOI: \href{https://doi.org/10.1086/670337}{10.1086/670337}

\bibitem{MOPS-KD-Tree}
Kubica, Jeremy et al.\\
\textit{Efficient intra- and inter-night linking of asteroid detections using kd-trees}\\
Icarus. July 2007.\\
Volume 189, Issue 1, Pages 151-168.\\
DOI: 	\href{https://doi.org/10.1016/j.icarus.2007.01.008}{10.1016/j.icarus.2007.01.008}\\
% arXiv:astro-ph/0703475

\bibitem{RL-Rebound}
Hanno Rein, Shang-Fei Liu\\
\textit{REBOUND: An open-source multi-purpose N-body code for collisional dynamics}.\\
Astronomy \& Astrophysics.  January 2012.\\
Volume 537, Number A128\\
DOI: \href{https://doi.org/10.1051/0004-6361/201118085}{10.1051/0004-6361/201118085}
% arXiv: 1110.4876v2

\bibitem{RL-IAS15} 
Hanno Rein, David S. Spiegel\\
\textit{IAS15: A fast, adaptive high-order integrator for gravitational dynamics, accurate to machine precision over a billion orbits}.\\
Monthly Notices of the Royal Astronomical Society.  11 January 2015.\\
Volume 446, Issue 2, Pages 1424-1437\\
DOI: \href{https://doi.org/10.1093/mnras/stu2164}{10.1093/mnras/stu2164}\\
% arXiv: 1405.4779.v2

\bibitem{SSD}
Murray, C. D.; Dermott, S.F. \\
\textit{Solar System Dynamics}\\
Cambridge University Press. 1999

\bibitem{MMCM}
Arnold, V.I.\\
\textit{Mathematical Methods of Classical Methods}\\
Springer-Verlag, Graduate Texts in Mathematics.  1991

\bibitem{BH}
Joseph Blitzstein, Jessica Hwang\\
\textit{Introduction to Probability}\\
CRC Press.  2019 (Second Edition)

\end{thebibliography}
\end{singlespacing}

% Appendices from all chapters should go at the end
% \input{appendix}

\end{document}
