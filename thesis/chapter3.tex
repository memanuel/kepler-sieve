\section{Introduction}
\label{section_ztf_intro}
Zwicky Transient Facility (\href{https://www.ztf.caltech.edu/}{ZTF}) is a time-domain survey of the northern sky
that had first light at Palomar Observatory in 2017.  It is run by CalTech.
My advisor Pavlos suggested it as a data source for this project.
The ZTF dataset has two major advantages for searching for asteroids:
\begin{itemize}
\item ZTF gives a wide and fast survey of the key, covering over 3750 square degrees an hours to a depth of 20.5 mag
\item A machine learning pipeline has been developed to classify a subset of ZTF detections that are classified as probable asteroids
\end{itemize}
The data set I analyze here consists of all ZTF detections that were classified as asteroids.
Data on each detection include:
\begin{itemize}
\item \textbf{ObjectID} an identifier of the likely ojbect associated with this detection; multiple detections often share the same ObjectID
\item \textbf{CandidateID} a unique integer identifier of each detection
\item \textbf{MJD} The time of the detection as an MJD
\item \textbf{RA} The right ascension of the detection
\item \textbf{Dec} The declination of the detection
\item \textbf{mag} The apparent magnitude of the detection
\end{itemize}
Available data also includes a number of additional fields that were not used in the analysis.

\href{https://github.com/alercebroker}{ALeRCE} (Automatic Learning for the Rapid Classification of Events) is an astronomical data broker.
ALeRCE provides a convenient API to access the ZTF asteroid data, which can be installed with \tty{pip}.
I used ALeRCE on this project to download the ZTF asteroid data set.

\section{Exploratory Data Analysis of ZTF Asteroid Data}
\label{section_ztf_eda}
Before plowing into the search for new asteroids, I conducted an exploratory data analysis (EDA) of the ZTF asteroid dataset.
This can be followed interactively in the Jupyter notebook \tty{05\_ztf\_data.ipynb}.
I took a download of the data running through 26-Feb-2020.
The first detection is on 01-Jun2018.
The dataset contains 5.69 total detections.  
The volume of detections increases very significantly beginning in July 2019; 
for practical purposes the dataset consists of 8 months of detections spanning July 2019 through February 2020.

\begin{figure}[hbt!]
\begin{center}
\includegraphics[width=1.0\textwidth]{../figs/ztf/ztf_ast_per_month.png}
\caption{ZTF Asteroid Detections per month}
\end{center}
\end{figure}

\begin{figure}[hbt!]
\begin{center}
\includegraphics[width=1.0\textwidth]{../figs/ztf/ztf_dataframe.png}
\caption{Preview of Pandas DataFrame of ZTF Detections}
\end{center}
\end{figure}

The fields \tty{mjd}, \tty{ra}, \tty{dec}, and \tty{mag\_app} are part of the original dataset.
I have populated the columns \tty{ux}, \tty{uy} and \tty{uz} by running \tty{radec2dir} on the quoted RA/Dec from ZTF.

\section{The Angular Distance Bewteen two Directions $\vec{u}_1$ and $\vec{u}_2$}
\label{section_ztf_angle_diff}
A recurring task in this thesis to compute the angular distance bewteen two directions on the unit sphere.
If $\uvec_1$ and $\uvec_2$ are on the unit sphere, we can compute their Cartesian distance $s$ in the usual way,
$$ s = \norm{\uvec_2 - \uvec_1}$$
$s$ will be in the interval $[0, 2]$.
We would also like to know the angular distance $\theta$ of the shortest path 
on the surface of the sphere (geodesic) connecting these two points.
On Earth, this would be analogous to the length of the great circle route by airplane between two cities.
Here is a simple picture demonstrating the derivation of the formula relating $s$ and $\theta$.
\begin{figure}[hbt!]
\begin{center}
\includegraphics[width=0.3\textwidth]{../figs/misc/angular_distance.png}
\caption{The angular distance between two directions on the unit sphere.}
\end{center}
\end{figure}
The two directions are shown as arrows pointing up.
The distance between them $s$ is bisected by a line segment from the center of the sphere.
This forms a right triangle with hypotenuse $1$ and side length $s/2$ opposite angle $\theta/2$.
We thus obtain the formulas
\begin{align*}
\sin (\theta / 2) &= s / 2 \\ 
\theta &= 2 \arcsin (s / 2) \\
s &= 2 \sin \theta (\theta / 2)
\end{align*}
This function is implemented in \tty{astro\_utils} as \tty{deg2dist} and \tty{dist2deg} to convert between degrees and Cartesian distance in either direction.

\section{Finding the Nearest Asteroid to Each ZTF Observation}
\label{section_ztf_nearest_ast}
We now have in principle all the tools required to find which asteroid was closest in angular distance to each ZTF observation.
To recap the key steps, each ZTF observation is converted from a RA/Dec to a direction in the BME.
The position of Earth and each of the 733,489 catalogued asteroids are integrated as of the objervation time, and the direction bewteen them is calculated.
There are a few problems with the brute force approach implicitly suggested above.
We have 5.7E6 detections and 7.3E5 catalogued asteroids for a total of 4.16E12 (4.16 billion) interactions.
Even if we work in single precision with 4 bytes per float, we need 12 bytes for a 3D direction difference translating to about 50 GB to load the matrix in memory.
The bigger problem is that a brute force integration of all the asteroids at the MJDs of the all 5.7 million observations would be brutally slow.

Fortunately there are a few simple tricks we can use that together make this problem tractable.
The ZTF detections come from a series of images taken through the same telescope, so they come in blocks made at the same time.
The 5.7 million rows share ``only'' 97,111 distinct MJDs.
We also don't need to re-integrate the asteroid orbits.
We've already done a high precision numerical integration at a 1 day frequency and saved the results into \tty{numpy} arrays in blocks of 1,000 asteroids at a time.
Our entire data span only 635 days.  
Loading one block of 1,000 asteroids therefore has only 3.81 million numbers (635 days x 1,000 asteroids x 6 numbers per integration point).

The code to load the basic ZTF data from ALeRCE is included in the module \tty{ztf\_data.py}.
The main function used by consumers is \tty{load\_ztf\_det\_all}, which loads a cached copy of all the available detections from the local disk.
The module \tty{ztf\_nearest\_ast.py} contains a Python program that peforms the calculation described above.
The block size is a parameter that can be controlled from the commandline; I ended up leaving it at 1,000.
Checking back from my handwritten notes when I ran the job, it took approximately 25 hours to complete on a powerful server with 40 Intel CPU cores.
The job ended up being memory bound, maxxing out all 256 GB of available RAM on the server.
The initial job described above writes only computes the nearest asteroid in a block of 1,000 asteroids.
A second reduction operation is then carried out to find the nearest asteroid overall.
To limit memory usage, this was done in two steps.
First, I took 16 chunks of 1,000 at a time to find the nearest asteroid in a block of 16,000 to each detection.
Then I combined all the blocks of 16,000 (about 46) to generate one file with the nearest asteroid.

The work of splining the asteroid directions is done in the module \tty{asteroid\_dataframe.py}.
It includes functions to load the asteroid data from disk; spline the positions and velocities to the requested dates;
and compute the astrometric directions to these splined positions and velocities.
The module \tty{ztf\_ast} does the work of comparing a block of ZTF observations to a block of splined asteroid directions and finding the nearest one.
Once these calculations have been done once, you don't need to worry about them unless you are adding a new block of ZTF data.
Consumers can load the assembled DataFrame including the nearest asteroid number and distance with 
a single call to \tty{load\_ztf\_nearest\_ast} which is defined in \tty{asteroid\_dataframe}.
For the motivated reader who would like a detailed an interactive review of all these calculations, please see the Jupyter notebook \tty{04\_asteroid\_dataframe.ipynb}.
It includes tests comparing my splined outputs of daily data to Horizons data downloaded at a 3 hour interval.

\begin{figure}[hbt!]
\begin{center}
\includegraphics[width=1.0\textwidth]{../figs/ztf/ztf_nearest_ast_dataframe.png}
\caption{Preview of Pandas DataFrame of ZTF Detections Including Nearest Asteroid}
\end{center}
\end{figure}

I would like to take a step back and review what has been presented thus far.
Over 4 billion interactions between a ZTF asteroid detection and the predicted position of a known asteroid in the sky have been generated.
These have been filtered to associate each ZTF detection with the known asteroid it is nearest to.
This is a powerful enrichment of the original ZTF dataset, and might open the door to some additional work in the future.
For example, it could be used to create a bulk data set linking the original image files to the asteroids they belong to.
This could in turn be used to refine the machine learning pipeline used to classify detections and guess when they belong to the same object.

\section{Analyzing the Distribution of Distance to the Nearest Asteroid}
\label{section_nearest_ast_distribution}
In this section I explore the statistical distribution of the Cartesian distance between observations and the nearest asteroid.
I compare the observed distribution of this distance to the theoretical distribution we would obtain 
if either our observed or predicted directions were distributed uniformly at random on the sphere.

\begin{figure}[hbt!]
\begin{center}
\includegraphics[width=0.36\textwidth]{../figs/misc/orange_slice.png}
\caption{Distance between an observation at the north pole and an arbitrary point.\\
The observed direction $\uobs$ is assumed w.l.o.g. to be at the north pole $(0, 0, 1)$.\\
The predicted direction $(x,y,z)$ is rotated in the $xy$ plane to to $(r, 0, z)$ \\
The Cartesian distance to from the observation to $(r,z)$ is $s$, the hypotenuse of a right triangle with sides $1-z$ and $r$.}
\end{center}
\end{figure}
Suppose without loss of generality that the observed direction is at the north pole, i.e. $\uobs = (0, 0, 1)$.
Suppose that the direction we predict is $(x, y, z)$.
Observe that the problem is symmetric about the $z$ axis, so we can rotate the problem into 
the plane containing the center, the north pole, and our guess.
Let $r^2 = x^2 + y^2$ be the squared distance from the $z$ axis to our guess.
This configuration is shown in the diagram above.
We can can relate the Cartesian distance $s$ to the height $z$ of our guess with the following simple observations.\\
$(x, y, z)$ lies on the surface of a sphere, so $x^2 + y^2 + z^2 = 1$.\\
$r^2 = x^2 + y^2$ by definition, so $r^2 = 1 - z^2$.\\
In our diagram, we can see that $s$ is the hypotenuse of a right triangle whose other two side have length $r$ and $1-z$.
Applying the Pythagorean Theorem, we find $s^2 = (1-z)^2 + r^s$. \\
This simplifies to
\begin{align*}
s^2 &= 2(1-z) \\
z &= 1 - \frac{s^2}{2}
\end{align*}
It turns out that this is a very useful parameterization because there is an elegant parameterization 
of the differential solid angle $d\Omega$ in terms of $dz$.
When surface integrals of a sphere are taught in introductory multivariate calculus courses, 
students are most likely to be exposed only to the surface element in spherical coordinates $(r, \theta, \phi)$ which is
$d\Omega = \sin \theta d\theta d\phi$.
See, e.g. \href{https://en.wikipedia.org/wiki/Spherical_coordinate_system}{Wikipedia - integration in spherical coordinates}.
But the parameterization in terms of the height $z$ along the $z$ axis is especially clean and convenient on this problem.\\
Observe that $z = \cos \theta$ so $dz = |- \sin \theta| d\theta = \sin \theta d\theta$.  
We take absolute values because this is an application of the change of variables formula and measures are always positive. \\
Substituting this expression for the surface element, we obtain
$$ d\Omega = dz d\phi$$
As a quick sanity check of this result, let's see if we can recover that the surface area of  the unit sphere is $4 \pi$:
$$ A = \int_{z = -1}^{1} \int_{\phi=0}^{2 \pi} d \phi dz = 2 \pi \int_{z = -1}^{1} dz = 4 \pi$$
We can now write the probability density function (PDF) for any function that can be expressed in terms of $z$.
Think of $Z$ as a random variable now.  The above result shows that when $Z$ is uniformly distributed on the unit sphere,
$$Z \sim \Unif(-1, 1)$$
Previously we showed that $z = 1 - s^2 / 2$.
This is a very useful result; it tells us that if we treat the squared distance as a random variable $S^2$, then it is uniformly distributed on $[0, 2]$.
Even more usefully, the conditional distribution of $S^2$, conditioned on $S^2 \le \tau^2$, is also uniform:
$$ S^2 | S^2 \le \tau^2 \sim \Unif(0, \tau^2)$$
If we apply a threshold distance $\tau$ and only consider predicted directions that are within Cartesian distance $\tau$ of an observation,
then the condtional distribution of the relative distance over the threshold squared is uniform on $[0, 1]$.
In mathematical notation instead of words, \\ 
Let $\upred$ be a random variable distributed uniformly on the unit sphere.\\
Let $S = \norm{\upred - \uobs}$ be the Cartesian distance between $\upred$ and $\uobs$. \\
Let $\tau$ be a threshold distance in [0, 2].\\
Let $V = S^2 / \tau^2$ be the relative squared distance of on observation vs. the threshold.\\
Then the conditional distribution of $V$, conditional on $S < \tau$ (equivalently $V < 1$) is
$$V \sim \Unif(0, 1)$$

% \section{Conclusion}
